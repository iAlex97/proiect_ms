\documentclass[12pt,english]{article}
\usepackage[a4paper,bindingoffset=0.2in,%
            left=0.4in,right=1in,top=1in,bottom=1in,%
            footskip=.25in]{geometry}
\usepackage{amsmath}
\usepackage{graphicx}
\graphicspath{ {./res/} }

\newcommand{\myparagraph}[1]{\paragraph{#1}\mbox{}\\}

\title{Tema 3}
\date{2018\\ Noiembrie}
\author{Ionescu Alexandru Cristian - 333 AC}

\begin{document}

\maketitle

\pagebreak


\myparagraph {Problema 1}

Consideram filtrul FIR de ordinul I:
\begin{center}
$\displaystyle H( z) \ =\ 1-cz^{-1} ,\ c\ =re^{j\theta } ,\ \theta \in [ 0,\ \pi ]$
\end{center}

\subparagraph {Pentru:\ $\displaystyle \theta \ =\ \pi /3$}

\begin{center}
\includegraphics[width=1\textwidth]{1_1.eps}
\end{center}

\begin{center}
\includegraphics[width=1\textwidth]{1_2.eps}
\end{center}

\begin{center}
\includegraphics[width=1\textwidth]{1_3.eps}
\end{center}

Pe diagrama poli-zerouri, se observa ca modulul si defazajul zeroului corespund cu valorile alese pentru $r \ si \ \theta$

\subparagraph {Pentru:\ $\displaystyle \theta \ =\ \pi /2$}

\begin{center}
\includegraphics[width=1\textwidth]{1_4.eps}
\end{center}

\begin{center}
\includegraphics[width=1\textwidth]{1_5.eps}
\end{center}

\begin{center}
\includegraphics[width=1\textwidth]{1_6.eps}
\end{center}

Pe diagrama poli-zerouri, se observa ca modulul si defazajul zeroului corespund cu valorile alese pentru $r \ si \ \theta$

\pagebreak
\myparagraph {Problema 2}

Consideram filtrul FIR de ordinul II:

\begin{center}
$\displaystyle H( z) \ =\ \left( 1-cz^{-1}\right)\left( 1-\overline{c} z^{-1}\right) \ =\ 1-2r*\cos( \theta ) z^{-1} +r^{2} z^{-2}$
\end{center}

\subparagraph {Pentru:\ $\displaystyle \theta \ =\ \pi /3$}

\begin{center}
\includegraphics[width=1\textwidth]{2_1.eps}
\end{center}

\begin{center}
\includegraphics[width=1\textwidth]{2_2.eps}
\end{center}

\begin{center}
\includegraphics[width=1\textwidth]{2_3.eps}
\end{center}

Pe diagrama poli-zerouri, se observa cele doua zerouri conjugate de modul egal cu $r$

\subparagraph {Pentru:\ $\displaystyle \theta \ =\ \pi /2$}

\begin{center}
\includegraphics[width=1\textwidth]{2_4.eps}
\end{center}

\begin{center}
\includegraphics[width=1\textwidth]{2_5.eps}
\end{center}

\begin{center}
\includegraphics[width=1\textwidth]{2_6.eps}
\end{center}

Pe diagrama poli-zerouri, se observa cele doua zerouri conjugate de modul egal cu $r$

\pagebreak
\myparagraph {Problema 3}

Fie filtrul autoregresiv $\displaystyle G( z) \ =\ \frac{1}{H( z)}$, cu functia de transfer:

\begin{center}
$\displaystyle G( z) \ =\ \frac{1}{\left( 1-cz^{-1}\right)\left( 1-\overline{c} z^{-1}\right) \ } =\ \frac{1}{1-2r*\cos( \theta ) z^{-1} +r^{2} z^{-2}}$
\end{center}

\subparagraph {Pentru:\ $\displaystyle \theta \ =\ \pi /3$}

\begin{center}
\includegraphics[width=1\textwidth]{3_1.eps}
\end{center}

Se observa oglindirea graficului amplitudinii lui $G(z)$ fata de $H(z)$ in raport cu abscisa.

\begin{center}
\includegraphics[width=1\textwidth]{3_2.eps}
\end{center}

\begin{center}
\includegraphics[width=1\textwidth]{3_3.eps}
\end{center}

Se observa ``inversarea'' polilor cu a zerourilor fata de problema anterioara.	

\subparagraph {Pentru:\ $\displaystyle \theta \ =\ \pi /3 \ si \ | G( 0)| _{dB} \ =\ 0$}

\begin{center}
\includegraphics[width=1\textwidth]{3_3b.eps}
\end{center}

\pagebreak
\subparagraph {Pentru:\ $\displaystyle \theta \ =\ \pi /2$}

\begin{center}
\includegraphics[width=1\textwidth]{3_4.eps}
\end{center}

Se observa oglindirea graficului amplitudinii lui $G(z)$ fata de $H(z)$ in raport cu abscisa.

\begin{center}
\includegraphics[width=1\textwidth]{3_5.eps}
\end{center}

\begin{center}
\includegraphics[width=1\textwidth]{3_6.eps}
\end{center}

Se observa ``inversarea'' polilor cu a zerourilor fata de problema anterioara.	

\subparagraph {Pentru:\ $\displaystyle \theta \ =\ \pi /2 \ si \ | G( 0)| _{dB} \ =\ 0$}

\begin{center}
\includegraphics[width=1\textwidth]{3_6b.eps}
\end{center}


\pagebreak
\myparagraph {Problema 4}

{Fie filtrul stabil $\displaystyle H( s) \ =\ \frac{s^{2} -0.85s+0.85}{s^{2} +0.65s+0.65}$ cu 2 poli si 2 zerouri.}

\subparagraph {Raspunsul filtrului in frecventa}
\begin{center}
\includegraphics[width=1\textwidth]{4_1.eps}
\end{center}

\subparagraph {Diagrama poli-zerouri}
\begin{center}
\includegraphics[width=1\textwidth]{4_2.eps}
\end{center}

Se observa cei doi poli si cele doua zerouri conjugate cu modulul mare mare ca 0.7

\pagebreak
\paragraph {Problema 5}

\subparagraph {Suma de sinusoide filtrata}
\begin{center}
\includegraphics[width=1\textwidth]{5a_1.eps}
\end{center}

\begin{center}
\includegraphics[width=1\textwidth]{5a_2.eps}
\end{center}
\pagebreak

\subparagraph {Vioara}
\begin{center}
\includegraphics[width=1\textwidth]{5a_3.eps}
\end{center}

\begin{center}
\includegraphics[width=1\textwidth]{5a_4.eps}
\end{center}
\pagebreak

\subparagraph {Viola}
\begin{center}
\includegraphics[width=1\textwidth]{5a_5.eps}
\end{center}

\begin{center}
\includegraphics[width=1\textwidth]{5a_6.eps}
\end{center}
\pagebreak

\subparagraph {Clarinet}
\begin{center}
\includegraphics[width=1\textwidth]{5a_7.eps}
\end{center}

\begin{center}
\includegraphics[width=1\textwidth]{5a_8.eps}
\end{center}
\pagebreak

\subparagraph {Trombon}
\begin{center}
\includegraphics[width=1\textwidth]{5a_9.eps}
\end{center}

\begin{center}
\includegraphics[width=1\textwidth]{5a_10.eps}
\end{center}
\pagebreak

\subparagraph {Tuba}
\begin{center}
\includegraphics[width=1\textwidth]{5a_11.eps}
\end{center}

\begin{center}
\includegraphics[width=1\textwidth]{5a_12.eps}
\end{center}
\pagebreak

\myparagraph {Trombon filtrat (1)}
Am aplicat filtrul: $\displaystyle H( s) \ =\ \frac{2s+1.5}{s^{2} -0.9s+0.5}$

\begin{center}
\includegraphics[width=1\textwidth]{5b_1.eps}
\end{center}

\begin{center}
\includegraphics[width=1\textwidth]{5b_2.eps}
\end{center}

\begin{center}
\includegraphics[width=1\textwidth]{5b_3.eps}
\end{center}
\pagebreak

\myparagraph {Trombon filtrat (2)}
Am aplicat filtrul: $\displaystyle H( s) \ =\ \frac{s^{2} -1.8s+0.9}{s^{3} -3s^{2} -2.4s+0.8}$

\begin{center}
\includegraphics[width=1\textwidth]{5b_4.eps}
\end{center}

\begin{center}
\includegraphics[width=1\textwidth]{5b_5.eps}
\end{center}

\begin{center}
\includegraphics[width=1\textwidth]{5b_6.eps}
\end{center}
\pagebreak

\myparagraph {Trombon filtrat (3)}
Am aplicat filtrul: $\displaystyle H( s) \ =\ \frac{1}{s-0.8}$

\begin{center}
\includegraphics[width=1\textwidth]{5b_7.eps}
\end{center}

\begin{center}
\includegraphics[width=1\textwidth]{5b_8.eps}
\end{center}

\begin{center}
\includegraphics[width=1\textwidth]{5b_9.eps}
\end{center}
\pagebreak

\end{document}