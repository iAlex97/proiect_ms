\documentclass[12pt,english]{article}
\usepackage[a4paper,bindingoffset=0.2in,%
            left=0.4in,right=1in,top=1in,bottom=1in,%
            footskip=.25in]{geometry}
\usepackage{amsmath}
\usepackage{graphicx}
\graphicspath{ {./res/} }

\newcommand{\myparagraph}[1]{\paragraph{#1}\mbox{}\\}

\title{Modelare și Simulare\\Temă laborator\\-\\Tema 2\\Instalație hidraulică cu patru rezervoare}
\date{2018\\Decembrie}
\author{Ionescu Alexandru Cristian\\Pangratie Andrei\\333 AC}

\begin{document}

\maketitle

\pagebreak


\myparagraph {Introducere}
Simularea procesului este salvata cu versiunea MATLAB R2017a si pentru rularea ei se va folosi scriptul ``run_tema_2017.m''. Pentru a rula cu MATLAB R2018a se va folosi scriptul run_tema.m

\myparagraph {Structura proiectului}
Ficare subpunct are doua fisiere aferente (load_workspace_*.m si tema_comm_*.m)
Fisierul ``animate_levels.m'' este folosit dupa fiecare simulare pentru a crea o reprezentare grafica a evolutiei nivelelor rezervoarelor cu apa.

Consideram filtrul FIR de ordinul I:
\begin{center}
$\displaystyle H( z) \ =\ 1-cz^{-1} ,\ c\ =re^{j\theta } ,\ \theta \in [ 0,\ \pi ]$
\end{center}

\subparagraph {Pentru:\ $\displaystyle \theta \ =\ \pi /3$}

% \begin{center}
% \includegraphics[width=1\textwidth]{1_1.eps}
% \end{center}

\end{document}